\documentclass [11pt, a4paper]{article}

\usepackage[left=1.5cm, text={18cm, 25cm}, top = 2.5cm]{geometry}
\usepackage{times}
\usepackage{amsmath}
\usepackage{amsthm}
\usepackage{amsfonts}
\usepackage[utf8]{inputenc}
\usepackage{graphicx}
\usepackage{float}
\usepackage[czech]{babel}
\usepackage{enumitem}
\usepackage{hyperref}
\setlength\parindent{0pt}

%\graphicspath{ {./images/} }
\begin{document}
% téma
\section*{Zvolené téma:} 
\setlist{nolistsep}
\begin{itemize}[noitemsep]
\item 04: COVID-19 (dr. Rýchly)
\end{itemize}


% členovia tímu
\subsection*{Riešitelia:}
\setlist{nolistsep}
\begin{itemize}[noitemsep]
\item Sabína Gregušová (xgregu02)
\item Peter Hamran (xhamra00)
\item Adrián Tulušák (xtulus00)
\end{itemize}

\section*{Úvod}
Cieľom tohto projektu bolo zoznámenie sa so spracovaním neštruktúrovaných dát, s~ich prípravou a~spracovaním pre ďalšie využitie. Následne vďaka nadobudnutým vedomostiam navrhnúť nástroje, ktoré umožnia automatické spracovanie a~nahranie dát do NoSQL databáze. Z~NoSQL databáze následne vybrať vhodné data, ktoré je po úprave možné uložiť do vhodne navrhnutej relačnej databázy. Poslednou časťou projektu je návrh a~implementácia prostredia na zodpovedanie dvoch zadaných dotazov a~jedného vlastného k~danej téme. 

Témy tohto projektu boli vopred zadané a~pre riešenie nášho projektu sme si vybrali tému COVID-19. Dáta využité na zodpovedanie dotazov pochádzajú z~verejne dostupných zdrojov. Na návrh systémov sme použili viacero voľne dostupných technológií a~programovacích jazykov, ktoré spolu vytvárajú celok. 
\section*{Príprava dát}

Tento projekt sa zameriava na situáciu v~Českej republike (dotazy~A,~B) a~na situáciu v~Éurópskej únii (dotaz C) od vypuknutia pandémie COVID-19. Keďže sa jedná o~stále aktuálne téma, existuje k~nemu pomerne veľké množstvo dát. Podarilo sa nám získať dátové sady, ktoré sú dostačujúce pre zodpovedanie všetkých troch dotazov. Jedná~sa~o:

\begin{itemize}
\item pre A: \href{https://onemocneni-aktualne.mzcr.cz/api/v2/covid-19/osoby.min.json }{COVID-19: Přehled osob s~prokázanou nákazou dle hlášení krajských hygienických stanic~(v2)}
\item pre B: \href{https://onemocneni-aktualne.mzcr.cz/api/v2/covid-19/kraj-okres-nakazeni-vyleceni-umrti.min.json}{COVID-19: Přehled epidemiologické situace dle hlášení krajských hygienických stanic podle okresu}
\item pre C: \href{https://www.ecdc.europa.eu/en/publications-data/covid-19-testing}{Testing for COVID-19 by week and country}
\end{itemize}

Tieto dátové sady obsahujú najpodstatnejšie informácie a~pre ich lepšiu prezentáciu boli vytvorené a~pridané 2 dátové sady. Dotazy by bolo možné zodpovedať aj bez nich, avšak v~tomto prípade ide o~zlepšenie čitatelnosti výsledku. Jedná~sa~o:
\begin{itemize}
\item \texttt{countries.json:} obsahuje unikátne skratky krajín a~ich český ekvivalent
\item \texttt{nuts\_lau.json:} obsahuje kódy NUTS (kraj) a~LAU (okres) a~ich český ekvivalent
\end{itemize} 

Dátové sady sú importované pomocou príkazu \texttt{mongoimport}. Dátové sady A~a~B~vyžadujú dodatočné formátovanie pre prácu s podstatnými dátami. Tento proces je automatizovaný pomocou skriptu \texttt{prepData.sh}, ktorý:
\begin{itemize}
\item Zmaže a~vytvorí novú (čistú) databázu s~názvom "corona"
\item Postupne stiahne, a~ak je potrebné aj naformátuje, dátové sady vo formáte JSON
\item Naimportuje každú dátovú sadu ako kolekciu; kolekcie sú pomenované podľa hlavnej dátovej sady (A,~B,~C) a~novo pridané dátové sady sú pomenované ako countries a~nuts\_lau
\end{itemize}

Samotnú úpravu, prípravu a~konverziu dát vykonáva skript \texttt{convertData.py}, ktorý sa spúšťa s~Python verziou~3. Tento skript bol navrhnutý tak, aby bol ľahko modifikovateľný a~rozšíriteľný, preto využíva princípy objektovo orientovaného návrhu. Na začiatku sa pripojí ku~NoSQL a MySQL databázi. Vytvorí relačnú databázu v~jazyku SQL - kód vytvorenia relačnej databáze sa nachádza v~samostatnom súbore s~názvom \texttt{Corona.sql}, ktorý je preparsovaný a~jednotlivé príkazy sú vykonané vrámci vrámci skriptu \texttt{convertData.py}. Toto riešenie zaisťuje, že samotný skript obsahuje minimum \uv{natvrdo} napísaných príkazov pre vytvorenie relačnej databáze.

Samotná relačná databáza obsahuje dokopy 6 tabuliek - 3 hlavné tabuľky pre zodpovedanie dotazov a~3~pomocné. Teoreticky by bolo možné použiť iba~3~hlavné tabuľky, ale mnohé dáta by boli redundantné a~neviedlo by to k~vhodnému návrhu. Slová ako \texttt{district} a~\texttt{region} sú často zameniteľné, no v~celom tomto projekte je slovo \texttt{district} považované za okres a~\texttt{region} za kraj.

\begin{figure}[H]
\begin{center}
\includegraphics[width=\linewidth]{relationalDatabase.png}
\caption{Schéma relačnej databázy}
\end{center}
\end{figure}

V skripte \texttt{convertData.py} sme navrhli všeobecnú funkciu \texttt{obtainImportantData}, ktorá chystá a upravuje všetky dáta. Táto funkcia dostane na vstupe dáta z MongoDB, zoznam záhlaví, ktoré chceme extrahovať a zoznam obmedzení pre jednotlivé záhlavia.\\

Demonštrácia použitia:\\
\texttt{self.obtainImportantData(C, ['population', 'country\_code'], ['', 'UPPER\_C'])}, \\kde C je kolekcia všetkých dát pre dotaz C z mongoDB, druhý parameter obsahuje zoznam záhlaví (dát), ktoré chceme v tomto prípade extrahovať (populácia a kód krajiny), a na záver je zoznam obmedzení rovnakej dĺžky ako bol zoznam záhlaví. Prázdna položka v zozname obmedzení značí "žiadne obmedzenie" a obmedzenie "UPPER\_C" prekonvertuje všetky kódy krajín do veľkých písmen. Vďaka tomuto návrhu spracovania dát je veľmi jednoduché modifikovať a rozširovať relačnúú databázu a samotnú úpravu dát.

\section*{Dotazy}
Dotazy sú v~projekte realizované pomocou mysql konektoru navrhnutého v~skriptovacom jazyku Python. Tento konektor je implementovaný v~súbore \texttt{query.py} a~využíva princípy objektovo orientovaného návrhu z~dôvôdu zvýšenej prehľadnosti a~zjednodušenie znovupoužiteľnosti. Takto navrhnutý objekt je následne využívaný na pozadí grafického rozhrania pre získavanie dát z~relačnej databázy.

Hlavný objekt obaluje inicializáciu spojenia s~relačnou databázou. Jednotlivé SQL doazy sú realizované formou metód tohto objektu, ktorých vstupy sú parametrami SQL dotazov a~výstupmi je hodnota alebo pole hodnôť na ktoré sa dotazujeme.

\section*{Grafické rozhranie}

\end{document}