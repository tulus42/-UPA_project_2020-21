\documentclass[11pt, a4paper]{article}

\usepackage[left=2cm, text={17cm, 24cm}, top = 3cm]{geometry}
\usepackage{times}
\usepackage[utf8]{inputenc}
\usepackage[T1]{fontenc}
\usepackage[czech]{babel}
\usepackage{enumitem}
\usepackage{hyperref}
\setlength\parindent{0pt}

\begin{document}
% téma
\section*{Zvolené téma:} 
\setlist{nolistsep}
\begin{itemize}[noitemsep]
\item 04: COVID-19 (dr. Rýchly)
\end{itemize}

% členovia tímu
\section*{Riešitelia:}
\setlist{nolistsep}
\begin{itemize}[noitemsep]
\item Sabína Gregušová (xgregu02)
\item Peter Hamran (xhamra00)
\item Adrián Tulušák (xtulus00)
\end{itemize}

% zvolené dotazy
\section*{Zvolené dotazy a formulácia vlastného dotazu:}
\setlist{nolistsep}
\begin{itemize}[noitemsep]
\item Dotaz A: popisné charakteristiky pre aspoň 4 údaje (dátum nahlásenia, pohlavie, vek, okres, zdroj nákazy) a prehľad osôb s prekázanou nákazou podľa hlásenia hygienických staníc 
\item Dotaz B: vliv počtu chorých a jeho zmeny v čase na susedné okresy, šírenie nákazy cez hranice okresov
\item Dotaz C (vlastný dotaz):  reálny percentuálny prírastok nakazených voči počtu testov v krajinách EU
\end{itemize}

% charakteristika dátovej sady pre každý dotaz
\section*{Charakteristika pre zvolené dátové sady}

\subsection*{Dotaz A}
Primárnym zdrojom je dátová sada KHS s názvom: \href{https://onemocneni-aktualne.mzcr.cz/api/v2/covid-19}{"COVID-19: Přehled osob s prokázanou nákazou dle hlášení krajských hygienických stanic (v2)"}, ktorá obsahuje dáta vo formáte JSON. Dáta obsahujú základné údaje o nakazenom: dátum hlásenia, vek, pohlavie, kód krajskej hygienickej stanice, kód okresu bydliska a či je nákaza importovaná zo zahraničia a ak áno, tak odkiaľ. Táto dátová sada obsahuje informácie o všetkých nakazených občanoch Českej republiky od počiatku pandémie.

\hfill \break
Tieto dáta taktiež umožnia agregované dotazy podľa dátumu nahlásenia, pohlavia, veku alebo okresu. Pri agregovaných dotazoch je možné rozšíriť celkový prehľad o priemerný vek nakazených, ktorý je často rozhodujúcim faktorom mortality nakazeného. Všetky dáta z tejto dátovej sady budú použité pre základný prehľad o nakazených a splnenie dotazu A.

\hfill \break
Dáta budú získané pomocou príkazu \texttt{curl} a importované cez príkaz \texttt{mongoimport} pomocou skriptu, ktorý takto pripraví celú databázu a jej kolekcie.
\subsection*{Dotaz B}
Primárnym zdrojom je dátová sada KHS s názvom: \href{https://onemocneni-aktualne.mzcr.cz/api/v2/covid-19}{"COVID-19: Přehled epidemiologické situace dle hlášení krajských hygienických stanic podle okresu"}, ktorá obsahuje dáta vo formáte JSON. Dáta obsahujú údaje o krajoch a okresoch a kumulatívne počty nakazených, vyliečených a úmrtí pre konkrétne dátumy.

\hfill \break
Dátová sada nám umožňuje zobraziť agregované údaje pre kraje, ale aj pre konkrétne okresy v ľubovolnom časovom rozmedzí. Keďže chceme sledovať iba šírenie nákazy, nebudeme využívať údaje o počte vyliečených a počte úmrtí.

\hfill \break
Dáta budú získané pomocou príkazu \texttt{curl} a importované cez príkaz \texttt{mongoimport} pomocou skriptu, ktorý takto pripraví celú databázu a jej kolekcie.

\subsection*{Dotaz C}
Zdrojom dát pre Dotaz C bude dátová sada dostupná na webovej lokalite ECDC (European Centre for Disease Prevention and Control) s názvom: \href{https://www.ecdc.europa.eu/en/publications-data/covid-19-testing}{"Testing for COVID-19 by week and country"}. Dátová sada obsahuje údaje vo formáte XLSX a pre nás zaujímavými sú počet nových prípadov a počet testov pre jednotlivé krajiny. Dátová sada ďalej obsahuje vypočítané percento nových prípadov k počtu testov. Údaje sú zozbierané z rôznych, verejne dostupných, zdrojov rôznymi spôsobmi. 

\hfill \break
Ďalej dátová sada obsahuje údaje ako spôsob získania dát, dvojmiestny kód krajiny, populáciu a množstvo testov na 100,000 obyvateľov. Tieto údaje nás avšak pre splnenie dotazu nezaujímajú a preto nie sú potrebné.

\hfill \break
Dátová sada dostupná z webovej lokality ECDC je natívne uložená vo formáte XLSX. Na konverziu dát do formátu JSON kvôli kompatibilite s MongoDB využijeme nástroje príkazového riadku Ubuntu linux. Pre konverziu z XLSX do formátu CSV využijeme nástroj \texttt{ssconvert} aplikácie \texttt{Gnumeric}. Následne takto konvertovaný CSV súbor preklopíme pomocou nástroja \texttt{csvtojson} do formátu JSON, ktorý dokáže databáza MongoDB prirodzene spracovať.

% popis NoSQL databázy
\section*{Zvolený spôsob uloženia surových dát}
Pre uloženie surových dát sme zvolili NoSQL databázu MongoDB, pretože má flexibilné schéma dokumentov a využíva štruktúru podobnú formátu JSON, ktorý sa dobre spracováva vo vysokoúrovňových jazykoch. Využitie formátu JSON bude teda na mieste, pretože väčšina dát, ktoré se zabostarali, je v tomto formáte už získavaná a zvyšné budú bez komplikácií konvertované.

\hfill \break
Databáza bude tvorená tromi kolekciami, každá z nich bude určená pre dátovú sadu jedného z dotazov (dáta popísané v časti dotaz A, B, C) a bude obsahovať všetky dáta získané z hore uvedených zdrojov. Dáta z kolekcií budú následne filtrované a agregované pre získanie potrebných údajov z databázy.
\end{document}